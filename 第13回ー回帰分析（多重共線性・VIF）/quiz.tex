\newcommand{\Release}{}
\newcommand{\Slide}{}
\newcommand{\PrintLecture}{1}
\newcommand{\PrintSolution}{1}
\newcommand{\MyCourse}{データサイエンスコース}
\newcommand{\MySemester}{春}
\newcommand{\MySubject}{ビジネス アナリティクス}
\newcommand{\MyClass}{第02回ーデータ収集}% フォルダ名自動挿入

%
% 科目共通定義
%

\newcommand{\OpenIntro}
{\MyRef{OpenIntro Statistics}{https://www.openintro.org/book/os}}

\newcommand{\ra}{\rightarrow}
\newcommand{\Ra}{\Rightarrow}

% Expectation E[X]
\def\E#1{E\big[#1\big]}
\def\S{\sum_{i=1}^n}

\newcommand{\B}{\hat{\beta}}
\newcommand{\SUM}{\sum_{i=1}^n}  % Summention from i=1 to n
\newcommand{\NH}{$\mathit{H}_0$} % Null hypthesis
\newcommand{\AH}{$\mathit{H}_1$} % Alternative hypothesis
\newcommand{\T}{\texorpdfstring{$t$}{}}% Student's t
\newcommand{\overtext}[3][1.5]{
  \mathrel{\overset{#2}{\scalebox{#1}[1]{$#3$}}}
}
\newcommand{\iid}{\overtext[2]{iid}{\sim}}
\newcommand{\convdist}{\overtext[2]{d}{\rightarrow}}
\newcommand{\convprob}{\overtext[2]{p}{\rightarrow}}
\newcommand{\as}[2]{\quad \text{as}\quad #1 \rightarrow #2}

\input{../../tex/hss_lualatex.tex}
\input{../../tex/hss_moodle.tex}
\input{../../tex/hss_beamer.tex}

\begin{document}

\maketitle

\section{小テスト(\MyClass)}

\begin{quiz}{\MyClass}

\QuizShortAnswer
{
  空欄を補充せよ。\\
  重回帰分析で相関関係が強い説明変数があると
  回帰係数の大きさが過大に推定される問題が発生する。
  この問題を\underline{     }
  (multi-collinearity)という。日本語では「マルチコ」とよく呼ばれる。
}
{
  重回帰分析で相関関係が強い説明変数があると
  回帰係数の大きさが過大に推定される問題が発生する。
  この問題を\MyFill{\ruby{多重共線性}{たじゅうきょうせんせい}}
  (multi-collinearity)という。日本語では「マルチコ」とよく呼ばれる。
}
{20}
{多重共線性}
{}
{}
{}

\QuizShortAnswer
{
  空欄を補充せよ。\\
  説明変数$x_i$を目的変数にした回帰モデル
  (ただし,$x_i$は説明変数から除く)
  $x_i = \beta_0 + \beta_1 x_1 + \cdots + \beta_p x_p$
  を用いて計算した決定係数を$R_i^2$としたとき,次式で表される指標を
  \underline{      }(VIF: variance inflation factor)という。
  \[VIF_i = \frac{1}{1-R_i^2}\]
  $VIF$は重回帰モデルを使った回帰分析における
  多重共線性の深刻さを定量化する指標として用いられる。
}
{
  説明変数$x_i$を目的変数にした回帰モデル
  (ただし,$x_i$は説明変数から除く)
  $x_i = \beta_0 + \beta_1 x_1 + \cdots + \beta_p x_p$
  を用いて計算した決定係数を$R_i^2$としたとき,次式で表される指標を
  \MyFill{分散拡大係数}(VIF: variance inflation factor)という。
  \[VIF_i = \frac{1}{1-R_i^2}\]
  $VIF$は重回帰モデルを使った回帰分析における
  多重共線性の深刻さを定量化する指標として用いられる。
}
{20}
{分散拡大係数}
{}
{}
{}

\QuizMultipleChoices
{
  説明変数$x_i$を目的変数にした回帰モデル
  (ただし,$x_i$は説明変数から除く)
  $x_i = \beta_0 + \beta_1 x_1 + \cdots + \beta_p x_p$
  を用いて計算した決定係数を$R_i^2$としたとき,
  $VIF$を表す数式として正しいものを選択せよ。
}
{
  $VIF_i = \frac{1}{1-R_i^2}$
}
{20}
{ $VIF_i = 1-\frac{1}{R_i}$}
{ $VIF_i = \frac{1}{1-R_i}$}
{ $VIF_i = 1-\frac{1}{R_i^2}$}
{*$VIF_i = \frac{1}{1-R_i^2}$}

\QuizShortAnswer
{
  空欄に入る数値を記入せよ。\\
  経験的に$VIF>$\underline{ }で多重共線性の程度が大きいと判断する。\\
  $VIF$が大きい説明変数をモデルから取り除くことで対処する。
}
{
  経験的に$VIF>10$で多重共線性の程度が大きいと判断する。\\
  $VIF$が大きい説明変数をモデルから取り除くことで対処する。
  \MyFig{0.6}{R/vif_r.png}
  \footnotesize
  $VIF = 10 \ra R = 0.95$,
  $VIF =  5 \ra R = 0.9$,
  $VIF =  3 \ra R = 0.8$
}
{40}
{10}
{}
{}
{}

\end{quiz}

\end{document}
