\newcommand{\Release}{}
\newcommand{\Slide}{}
\newcommand{\PrintLecture}{1}
\newcommand{\PrintSolution}{1}
\newcommand{\MyCourse}{データサイエンスコース}
\newcommand{\MySemester}{春}
\newcommand{\MySubject}{ビジネス アナリティクス}
\newcommand{\MyClass}{第02回ーデータ収集}% フォルダ名自動挿入

%
% 科目共通定義
%

\newcommand{\OpenIntro}
{\MyRef{OpenIntro Statistics}{https://www.openintro.org/book/os}}

\newcommand{\ra}{\rightarrow}
\newcommand{\Ra}{\Rightarrow}

% Expectation E[X]
\def\E#1{E\big[#1\big]}
\def\S{\sum_{i=1}^n}

\newcommand{\B}{\hat{\beta}}
\newcommand{\SUM}{\sum_{i=1}^n}  % Summention from i=1 to n
\newcommand{\NH}{$\mathit{H}_0$} % Null hypthesis
\newcommand{\AH}{$\mathit{H}_1$} % Alternative hypothesis
\newcommand{\T}{\texorpdfstring{$t$}{}}% Student's t
\newcommand{\overtext}[3][1.5]{
  \mathrel{\overset{#2}{\scalebox{#1}[1]{$#3$}}}
}
\newcommand{\iid}{\overtext[2]{iid}{\sim}}
\newcommand{\convdist}{\overtext[2]{d}{\rightarrow}}
\newcommand{\convprob}{\overtext[2]{p}{\rightarrow}}
\newcommand{\as}[2]{\quad \text{as}\quad #1 \rightarrow #2}

\input{../../tex/hss_lualatex.tex}
\input{../../tex/hss_hyperref.tex}
\input{../../tex/hss_beamer.tex}

\setbeameroption{hide notes}
%\setbeameroption{show notes}
%\setbeameroption{show only notes}
%\setbeameroption{show notes on second screen=right}

\begin{document}

\maketitle

\MyFrame{}{\tableofcontents}

\section{表データの名称}

\MyFrame{表データの名称}
{
  \begin{description}[レコード(record/row)]
    \item[テーブル(table)]データを表形式にまとめたもの
    \item[カラム(column)]列方向のデータ
    \item[レコード(record/row)]行方向のデータ
    \item[フィールド(field)]データの要素(Excelではセルとよぶ)
    \item[カラム名/項目名]表の一番上のカラムの内容を表記したもの
  \end{description}
  \MyFig{0.5}{table_names}
  \MyRef
  {データベースの用語を理解しよう 「テーブル」「レコード」「カラム」「フィールド」とは?}
  {https://academy.gmocloud.com/know/20160425/2259}
}

\section{ヒストグラム}

\MyFrame{}
{
  \MyDefinition{度数分布表}
  {
    データの値の範囲を小分けにした区間(bin)を
    \MyFill{階級}(class)という.
    各階級について,区間の真ん中の値を
    \MyFill{階級値}(class value),
    含まれるデータの個数のことを
    \MyFill{度数}(frequency/count)という.
    度数分布表における\MyFill{最頻値}(mode)は,
    最も度数の大きい階級値のことをいう.
    階級と度数を表にしたものを
    \MyFill{度数分布表}(FDT; frequency distribution table)という.
    すべての階級の度数を合計するとデータサイズに一致する.
  }
  \MyDefinition{ヒストグラム}
  {
    度数分布表の階級を横軸に,度数を縦軸にとった棒グラフを
    \MyFill{ヒストグラム}(histogram)/\MyFill{度数分布図}という.
  }
}

\MyFrame{}
{
  \MyFig{1.0}{frequency_table.pdf}
  \MyRef
  {【統計学の基礎知識】度数分布やヒストグラムについて端的に解説 }
  {https://korekara-marketing.com/statistics-frequency-distribution/}
}

\MyFrame{度数分布表からヒストグラムを作成}
{
  \MyFig{1.0}{freqtable2histogram.pdf}
  \MyRef
  {【統計学の基礎知識】度数分布やヒストグラムについて端的に解説 }
  {https://korekara-marketing.com/statistics-frequency-distribution/}
}

\MyFrame{ヒストグラム(複数グループ)}
{
  \MyFig{1.0}{histogram_three}
}

\MyFrame{ヒストグラムの階級幅}
{
  データサイズから最適な階級幅を推定する方法が複数考案されている.
  しかし,最適な幅はデータの分布形状に強く依存するため,
  通常,データの分布の特徴が最も良く現れるように,
  試行錯誤で階級幅を変えることが必要である.
}

\MyFrame{}
{
  \MyFig{1.0}{histogram_bin_choices.pdf}
  \MyRef
  {【ウェイキペディア】ヒストグラム}
  {https://ja.wikipedia.org/wiki/ヒストグラム}
}

\MyFrame{ヒストグラムの形状}
{
  \MyFig{0.9}{histgram_pattern.pdf}
  \MyRef
  {【なるほど統計学園】ヒストグラムの形状}
  {https://www.stat.go.jp/naruhodo/4_graph/shokyu/histogram.html}
}
\note
{
  ヒストグラムはデータの分布をみるのに有効なグラフです.
  データの分布をヒストグラムで表すことによって,
  そのデータの集合の特徴を把握することができます.
}

\MyFrame{分布形状を表現する用語}
{
  【ピーク】
  \MyItems
  {
   \item 単峰性(unimodal)/釣鐘状(bell-shaped)・・・山が一つ
   \item 双峰性(bimodal)・・・山が二つ
   \item 多峰性(multimodal)・・・山が三つ以上
   \item 一様(uniform)・・・山になっていなくて台地,フラット
   \item ランダム(random)・・・凸凹無秩序
  }
}

\MyFrame{分布形状を表現する用語}
{
【左右への歪み】
  \MyItems
  {
   \item 右歪(skewed right)/正歪(positively skewed)\\
     ・・・山の右側の裾野が広い
   \item 左歪(skewed left)/負歪(negatively skewed)\\
     ・・・山の左側の裾野が広い
   \item 左右対称(symmetric)・・・山の左右が同じ
  }
}

\MyFrame{分布形状を表現する用語}
{
【広がり】
  \MyItems
  {
   \item 裾野の広い/幅広の(wide spread)・・・なだらかな山
   \item 尖った/狭い(narrow spread)・・・ 急峻な山
  }
  \begin{itembox}{使用例}
    このデータは幅広の単峰性の左右対称の分布を持つ.\\
    このデータは尖った単峰性のやや右歪みのある分布を持つ.\\
    このデータは一様な分布を持つ.
  \end{itembox}
}

\MyFrame{分布形状を表現する用語}
{
  \MyProblem
  {
    次の分布の特徴を適切に文書で記せ.
    \MyFig{0.9}{problem_histogram_types.png}
  }
}

\MyFrame{分布形状を表現する用語}
{
  \MySolution
  {
    (省略)
  }
}

\section{ヒストグラム 問題演習}

\MyFrame{\insertsection}
{
  \MyProblem
  {
    \MyFig{1.0}{quiz/problem21.pdf}
  }
  \vspace{40mm}
}

\MyFrame{\insertsection}
{
  \MySolution
  {
    \MyFig{0.8}{quiz/solution21.pdf}
  }
}

\MyFrame{\insertsection}
{
  \MyProblem
  {
    \MyFig{0.9}{quiz/problem37.pdf}
    \MyFig{0.7}{quiz/problem37graph.pdf}
  }
  \vspace{40mm}
}

\MyFrame{\insertsection}
{
  \MySolution
  {
    \MyFig{0.8}{quiz/solution37.pdf}
  }
}

\MyFrame{\insertsection}
{
  \MyProblem
  {
    \MyFig{0.8}{quiz/problem46.pdf}
  }
  \vspace{40mm}
}

\MyFrame{\insertsection}
{
  \MySolution
  {
    \MyFig{0.8}{quiz/solution46.pdf}
  }
}

\MyFrame{\insertsection}
{
  \MyProblem
  {
    \MyFig{0.8}{quiz/problem45.pdf}
  }
  \vspace{40mm}
}

\MyFrame{\insertsection}
{
  \MySolution
  {
    \MyFig{0.8}{quiz/solution45.pdf}
  }
}

\MyFrame{\insertsection}
{
  \MyProblem
  {
    \MyFig{0.9}{quiz/problem26.pdf}
    \MyFig{0.9}{quiz/problem26graph.pdf}
  }
  \vspace{40mm}
}

\MyFrame{\insertsection}
{
  \MySolution
  {
    \MyFig{0.9}{quiz/solution26.pdf}
  }
}

\MyFrame{\insertsection}
{
  \MyProblem
  {
    \MyFig{0.9}{quiz/problem22.pdf}
    \MyFig{0.9}{quiz/problem22graph.pdf}
  }
  \vspace{40mm}
}

\MyFrame{\insertsection}
{
  \MySolution
  {
    \MyFig{0.9}{quiz/solution22.pdf}
  }
}

\MyFrame{\insertsection}
{
  \MyProblem
  {
    \MyFig{1.0}{quiz/problem30.pdf}
  }
  \vspace{40mm}
}

\MyFrame{\insertsection}
{
  \MySolution
  {
    \MyFig{0.6}{quiz/solution30.pdf}
  }
}

\section{ヒストグラム R演習}

\MyFrame{\insertsection}
{
  RStudio Cloudで,
  次のURLにあるソースコードを\red{タイプ}しヒストグラムを作成せよ.
  \alert{コピペすると記憶に定着しないためNG}
  \url{https://rpubs.com/tkdhss111/histogram}
}

\MyFrame{}
{
  \MyFig{0.9}{fig/histogram_data.png}

}

\MyFrame{R カラーパレット}
{
  \MyFig{0.8}{R_color_palette.png}
  \MyFig{0.7}{RGB.png}
  \MyRef{【ウィキペディア】RGB}{https://ja.wikipedia.org/wiki/RGB}
}

\MyFrame{}
{
  \MyFig{0.8}{fig/histogram_src.png}

}

\MyFrame{}
{
  \MyFig{1.0}{fig/histogram_graph.png}

}

\MyFrame{ヒストグラム R課題}
{
  次のデータを用いてRでヒストグラムを作成せよ.
  また,\textbf{hist}関数のオプション\textbf{breaks}を
  "Sturges"(default), "Scott", "FD"に変更して
  階級幅の自動推定方法の違いを確認せよ.\\
  \MyFig{0.9}{quiz/hw.png}
}

\MyFrame{ヒストグラム R課題}
{
  ヒストグラムが完成したら,
  RStudioの右上のアイコン
  \includegraphics[width=8mm]{icon_publish.pdf}をクリックして,
  RPubsの入力画面で次の内容を入力し,
  www公開(publish)せよ.\\
  \alert{www公開されるので個人名など機微な情報は入力しないこと.}
  \begin{description}
    \item[Username] tiu学籍番号 (例)tiu22110001\\
    \item[Title] ヒストグラム\\
    \item[Description] (空白/説明を入れてもよい)\\
    \item[Slug] histogram
  \end{description}
  (職場で自分のソースコードとしてすぐに活用できるように,
    自分なりの補足説明やコメントを入れておきましょう.)\\
    ここでの方法に追加して,
    他のRのパッケージを使って作図したら評点に加えます.
}

\MyFrame{最近人気のRパッケージ: \textbf{ggplot2}}
{
  美しいグラフを簡単に作成できるパッケージ.\\
  (構文がパイプ方式でやや特殊)
  \MyFig{1.0}{ggplot2.png}
  \MyRef{【ggplot2】gallery}{https://exts.ggplot2.tidyverse.org/gallery}
}

\end{document}
