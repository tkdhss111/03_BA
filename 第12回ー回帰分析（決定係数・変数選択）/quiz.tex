\newcommand{\Release}{}
\newcommand{\Slide}{}
\newcommand{\PrintLecture}{1}
\newcommand{\PrintSolution}{1}
\newcommand{\MyCourse}{データサイエンスコース}
\newcommand{\MySemester}{春}
\newcommand{\MySubject}{ビジネス アナリティクス}
\newcommand{\MyClass}{第02回ーデータ収集}% フォルダ名自動挿入

%
% 科目共通定義
%

\newcommand{\OpenIntro}
{\MyRef{OpenIntro Statistics}{https://www.openintro.org/book/os}}

\newcommand{\ra}{\rightarrow}
\newcommand{\Ra}{\Rightarrow}

% Expectation E[X]
\def\E#1{E\big[#1\big]}
\def\S{\sum_{i=1}^n}

\newcommand{\B}{\hat{\beta}}
\newcommand{\SUM}{\sum_{i=1}^n}  % Summention from i=1 to n
\newcommand{\NH}{$\mathit{H}_0$} % Null hypthesis
\newcommand{\AH}{$\mathit{H}_1$} % Alternative hypothesis
\newcommand{\T}{\texorpdfstring{$t$}{}}% Student's t
\newcommand{\overtext}[3][1.5]{
  \mathrel{\overset{#2}{\scalebox{#1}[1]{$#3$}}}
}
\newcommand{\iid}{\overtext[2]{iid}{\sim}}
\newcommand{\convdist}{\overtext[2]{d}{\rightarrow}}
\newcommand{\convprob}{\overtext[2]{p}{\rightarrow}}
\newcommand{\as}[2]{\quad \text{as}\quad #1 \rightarrow #2}

\input{../../tex/hss_lualatex.tex}
\input{../../tex/hss_moodle.tex}
\input{../../tex/hss_beamer.tex}

\begin{document}

\maketitle

\section{小テスト(\MyClass)}

\begin{quiz}{\MyClass}

\QuizMultipleChoices
{
  決定係数を表す式を選択せよ。
}
{
  $R^2=\frac{回帰変動}{全変動}$
}
{20}
{*$R^2=\frac{回帰変動}{全変動}$}
{ $R^2=\frac{回帰変動}{残差変動}$}
{ $R^2=\frac{残差変動}{全変動}$}
{ $R^2=1-\frac{回帰変動}{全変動}$}

\QuizMultipleChoices
{
  全変動,残差変動をそれぞれの自由度割った決定係数を
  \underline{           }という。$adj.R^2$や$R_f^2$と表記される。
}
{
  全変動,残差変動をそれぞれの自由度割った決定係数を
  \MyFill{自由度調整済み決定係数}という。$adj.R^2$や$R_f^2$と表記される。
  \[adj.R^2
    =1-\frac{残差変動/残差変動の自由度}{全変動/全変動の自由度}
    =1-\frac{\frac{\sum(y_i-\hat{y})^2}{n-p-1}}
            {\frac{\sum(y_i-\bar{y})^2}{n-1}}\]
  ここで,$n,p$はそれぞれ,標本サイズ,説明変数の数である。
}
{20}
{ 自由な決定係数}
{*自由度調整済み決定係数}
{ 自由度調整不要決定係数}
{ 自由度未調整決定係数}

\QuizMultipleChoices
{
  統計数理研究所の赤池弘次所長が考案した情報量基準
  を英語三文字で何というか?
  また,予測モデルの評価ではこの値は大きい方が良いか否か?
}
{
  統計数理研究所の赤池弘次所長が考案した情報量基準
  AIC(Akaike Information Criterion)\\
  \[AIC=-2L+2p\]
  ここで,
  $L$:モデルの最大\ruby{対数尤度}{たいすうゆうど},
  $p$:モデルの説明変数の数\\
  AICは値が小さければ小さいほど良い予測モデルとなる。
  \ruby{交差検証法}{こうさけんしょうほう}で評価しているのと漸近的に等価である。
}
{20}
{*AIC 値が小さいほど良い予測モデル}
{ BIC 値が小さいほど良い予測モデル}
{ AIC 値が大きいほど良い予測モデル}
{ BIC 値が大きいほど良い予測モデル}

\QuizShortAnswer
{
  空欄を補充せよ。\\
  回帰分析に使用する説明変数の数$p$に比べて
  標本サイズ$n$が著しく少ない場合に
  \underline{   }と呼ばれる問題が発生する。
}
{
  回帰分析に使用する説明変数の数$p$に比べて
  標本サイズ$n$が著しく少ない場合に,次に示す
  \MyFill{過学習}と呼ばれる問題が発生する。
  この問題を回避するには,回帰係数の数($p+1$)の
  \MyFill{10倍}以上の標本サイズ(学習データサイズ)
  $n\ge10(p+1)$が必要とされている(経験則)。
}
{20}
{過学習}
{}
{}
{}

\QuizShortAnswer
{
  空欄を補充せよ。\\
  変数選択法の一つである逐次選択法では,増加法,減少法,\underline{   }などがある。
}
{
  増減法(ステップワイズ法; stepwise method)
}
{20}
{増減法}
{}
{}
{}


\end{quiz}

\end{document}
