\newcommand{\Release}{}
\newcommand{\Slide}{}
\newcommand{\PrintLecture}{1}
\newcommand{\PrintSolution}{1}
\newcommand{\MyCourse}{データサイエンスコース}
\newcommand{\MySemester}{春}
\newcommand{\MySubject}{ビジネス アナリティクス}
\newcommand{\MyClass}{第17回ー分類}% フォルダ名自動挿入

%
% 科目共通定義
%

\newcommand{\OpenIntro}
{\MyRef{OpenIntro Statistics}{https://www.openintro.org/book/os}}

\newcommand{\R}{\textbf{R}}
\newcommand{\RStudio}{\textbf{RStudio}}
\newcommand{\Excel}{\textbf{Excel}}
\newcommand{\cs}[1]{\textcolor{blue}{\texttt{#1}}} % Console prompt >

\newcommand{\ra}{\rightarrow}
\newcommand{\Ra}{\Rightarrow}

% Expectation E[X]
\def\E#1{E\big[#1\big]}
\def\S{\sum_{i=1}^n}

\newcommand{\B}{\hat{\beta}}
\newcommand{\SUM}{\sum_{i=1}^n}  % Summention from i=1 to n
\newcommand{\NH}{$\mathit{H}_0$} % Null hypthesis
\newcommand{\AH}{$\mathit{H}_1$} % Alternative hypothesis
\newcommand{\T}{\texorpdfstring{$t$}{}}% Student's t
\newcommand{\overtext}[3][1.5]{
  \mathrel{\overset{#2}{\scalebox{#1}[1]{$#3$}}}
}
\newcommand{\iid}{\overtext[2]{iid}{\sim}}
\newcommand{\convdist}{\overtext[2]{d}{\rightarrow}}
\newcommand{\convprob}{\overtext[2]{p}{\rightarrow}}
\newcommand{\as}[2]{\quad \text{as}\quad #1 \rightarrow #2}

\input{../../tex/hss_lualatex.tex}
\input{../../tex/hss_moodle.tex}
\input{../../tex/hss_beamer.tex}

\begin{document}

\section{小テスト(\MyClass)}

\begin{quiz}{\MyClass}

\QuizTrueFalse
{「強い正または負の相関係数が得られたとしても,
サンプルサイズが小さいときは信頼出来ない場合がある」は正しいか?}
{正しい.無相関検定で有意でなければ相関について評価できない.}
{30}
{*正しい}
{正しくない}

\QuizMultipleChoices
{無相関検定で有意とならなかった場合,誤っている表現はどれか?}
{
  今回のデータからは相関があるという証拠は見つからなかったということ.
  統計的仮説検定では,積極的に無相関と言うことはできない.
}
{40}
{相関があるとはいえない}
{もう一度同じ母集団からデータをとって検定すれば相関があるといえるかもしれない.}
{*無相関である}
{有意な相関はない}

\QuizShortAnswer
{
    二つの事象に因果関係が無いにも関わらず,
    他の要因によってあたかも因果関係があるかのように見える
    相関のことを\MyFill{    }という.空欄を補充せよ.
}
{
  擬似相関,または見せかけの相関という.
}
{30}
{擬似相関}
{見せかけの相関}
{spurious correlation}
{Spurious correlation}

\end{quiz}

\end{document}
