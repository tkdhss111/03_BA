\newcommand{\Release}{}
\newcommand{\Slide}{}
\newcommand{\PrintLecture}{1}
\newcommand{\PrintSolution}{1}
\newcommand{\MyCourse}{データサイエンスコース}
\newcommand{\MySemester}{春}
\newcommand{\MySubject}{ビジネス アナリティクス}
\newcommand{\MyClass}{第02回ーデータ収集}% フォルダ名自動挿入

%
% 科目共通定義
%

\newcommand{\OpenIntro}
{\MyRef{OpenIntro Statistics}{https://www.openintro.org/book/os}}

\newcommand{\ra}{\rightarrow}
\newcommand{\Ra}{\Rightarrow}

% Expectation E[X]
\def\E#1{E\big[#1\big]}
\def\S{\sum_{i=1}^n}

\newcommand{\B}{\hat{\beta}}
\newcommand{\SUM}{\sum_{i=1}^n}  % Summention from i=1 to n
\newcommand{\NH}{$\mathit{H}_0$} % Null hypthesis
\newcommand{\AH}{$\mathit{H}_1$} % Alternative hypothesis
\newcommand{\T}{\texorpdfstring{$t$}{}}% Student's t
\newcommand{\overtext}[3][1.5]{
  \mathrel{\overset{#2}{\scalebox{#1}[1]{$#3$}}}
}
\newcommand{\iid}{\overtext[2]{iid}{\sim}}
\newcommand{\convdist}{\overtext[2]{d}{\rightarrow}}
\newcommand{\convprob}{\overtext[2]{p}{\rightarrow}}
\newcommand{\as}[2]{\quad \text{as}\quad #1 \rightarrow #2}

\input{../../tex/hss_lualatex.tex}
\input{../../tex/hss_hyperref.tex}
\input{../../tex/hss_beamer.tex}

\setbeameroption{hide notes}
%\setbeameroption{show notes}
%\setbeameroption{show only notes}
%\setbeameroption{show notes on second screen=right}

\begin{document}

\maketitle

\MyFrame{}{\tableofcontents}

\note
{
  回帰分析は,今日最もよく使われている統計分析手法の一つ
  相関分析,回帰分析について,
  他の人に説明が出来るようになります.
  また,回帰分析に欠かせない最小二乗法の仕組みを理解できます.
}

\section{相関分析と回帰分析の違い}

\MyFrame{\insertsection}
{
  \MyTbl{lll}
  {
    分析名 & 変数 & 分析目的 \\
    \midrule
    回帰分析 & 目的変数:1, 説明変数:2~ & 説明変数$\ra$目的変数への影響\\
    相関分析 & 1 対 1 & 変数間の関係性(双方向)
  }
  相関分析と回帰分析はとても似た分析であるが,
  変数の種類や分析の目的がそれぞれ異なる.
  変数間の関係性や影響の強さが確認されても,
  変数間に因果関係があるとは限らない.
}
%因果関係
%https://bodais-datascientist.blogspot.com/2017/04/blog-post_25.html

\section{単回帰モデル}

\subsection{切片なし単回帰モデル}

\MyFrame{\insertsubsection}
{
  \[Y_i = \beta x_i + E_i, \quad E_i \sim \mathbb{N}(0, \sigma^2)\]
  ここで,\\
  \begin{itemize}
    \item [$Y_i$] 目的変数(\MyFill{確率変数})
    \item [$x_i$] 説明変数(変数)
    \item [$E_i$] 誤差項(\MyFill{確率変数})
    \item [$\beta$] 傾きを表す回帰係数(定数)
    \item [$\sigma$] 誤差の標準偏差(定数)
  \end{itemize}
}

\subsection{(切片あり)単回帰モデル}

\MyFrame{\insertsubsection}
{
  \[Y_i = \beta_0 + \beta_1 x_i + E_i, \quad E_i \sim \mathbb{N}(0, \sigma^2)\]
  ここで,\\
  \begin{itemize}
    \item [$Y_i$] 目的変数(\MyFill{確率変数})
    \item [$x_i$] 説明変数(変数)
    \item [$E_i$] 誤差項(\MyFill{確率変数})
    \item [$\beta_0$] 切片を表す回帰係数(定数)
    \item [$\beta_1$] 傾きを表す回帰係数(定数)
    \item [$\sigma$] 誤差の標準偏差(定数)
  \end{itemize}
  \alert{単に単回帰モデルと言うときは切片ありのモデルになる.}
}

\subsection{用語の定義}

\MyFrame{\insertsubsection}
{
  \MyDefinition{回帰係数の推定量}
  {
    $\beta$の推定量を$\hat{\beta}$で表す.
    「beta hat」と読む.
    $\hat{\beta}$は正規分布に従う確率変数であり,
    $E[\hat{\beta}]=\beta$となる不偏推定量である.
    $\hat{\beta}$の点推定値(実現値)を$b$で表す.
  }
  誤差項の$E_i$(epsilonの大文字)も正規分布に従う確率変数であり,
  その実現値を$e_i$で表す.
}

\section{最小二乗法}

\MyFrame{データの特徴を表す最も良い直線はどれか?}
{
  \begin{minipage}{0.3\textwidth}
    \MyFig{1.0}{line01.png}
  \end{minipage}
  \begin{minipage}{0.3\textwidth}
    \MyFig{1.0}{line02.png}
  \end{minipage}
  \begin{minipage}{0.3\textwidth}
    \MyFig{1.0}{line03.png}
  \end{minipage}
  \begin{minipage}{0.3\textwidth}
    \MyFig{1.0}{line04.png}
  \end{minipage}
  \begin{minipage}{0.3\textwidth}
    \MyFig{1.0}{line05.png}
  \end{minipage}
  \begin{minipage}{0.3\textwidth}
    \MyFig{1.0}{line06.png}
  \end{minipage}
}

\MyFrame{}
{
  観測値$y_i$とモデル(データの特徴を表した関数)
  による推定値$\hat{y}_i$との差を
  \MyFill{残差}(residual)といい$e_i$で表す.\\
  残差平方和$\sum_{i=1}^{10} e_i^2$
  の値が小さい~\ra~データに良く当てはまる直線\\[3mm]
  %
  \begin{minipage}{0.3\textwidth}
    \MyFig{1.0}{line_ss01.png}
  \end{minipage}
  \begin{minipage}{0.3\textwidth}
    \MyFig{1.0}{line_ss02.png}
  \end{minipage}
  \begin{minipage}{0.3\textwidth}
    \MyFig{1.0}{line_ss03.png}
  \end{minipage}
  \begin{minipage}{0.3\textwidth}
    \MyFig{1.0}{line_ss04.png}
  \end{minipage}
  \begin{minipage}{0.3\textwidth}
    \MyFig{1.0}{line_ss05.png}
  \end{minipage}
  \begin{minipage}{0.3\textwidth}
    \MyFig{1.0}{line_ss06.png}
  \end{minipage}
}

\MyFrame{計算で求まる最適な直線}
{
  \MyFig{0.8}{ls_line.png}
}

\MyFrame{}
{
  \MyDefinition{最小二乗法(最小自乗法)}
  {
    観測値$y_i$とモデル(関数)による推定値
    $\hat{y}_i=f(x_i)$との差である
    残差$e_i$の平方和(残差平方和; sum of squared errors)$SSE$
    を最小にする関数の係数を求める方法を
    \MyFill{最小二乗法}(LS; least squares method)という .
    \[
      SSE=\sum_{i=1}^n e_i^2
         =\sum_{i=1}^n\big\{y_i-\hat{y}_i\big\}^2
    \]
  }
  \MyExample{切片\red{なし}の単回帰モデルの場合}
  {
    観測値$y_i$,推定値$\hat{y}_i=b x_i$
    ,残差を$e_i$とすると,
    \[
      SSE=\sum_{i=1}^n e_i^2
         =\sum_{i=1}^n\big\{y_i-\hat{y}_i\big\}^2
         =\sum_{i=1}^n\big\{y_i-b x_i\big\}^2
    \]
    $SSE$を最小にする$b$を求める.
  }
}
\note{note\\note outside of frame}

\MyFrame{\insertsection}
{
  \MyExample{切片\red{あり}の単回帰モデルの場合}
  {
    観測値$y_i$,推定値$\hat{y}_i=b_0+b_1 x_i$
    ,残差を$e_i$とすると,
    \[
      SSE=\sum_{i=1}^n e_i^2
         =\sum_{i=1}^n\big\{y_i-\hat{y}_i\big\}^2
         =\sum_{i=1}^n\big\{y_i-(b_0+b_1 x_i)\big\}^2
    \]
    $SSE$を最小にする$b_0,b_1$を求める.
  }
}

\MyFrame{回帰係数の推定 (切片\red{なし}単回帰モデル)}
{
  切片なし単回帰モデル($Y_i=\beta x_i+E_i$)
  の残差平方和$SSE(\B)$を$\B$で偏微分し,
  $SSE(\B)$を最小にする$\B$を求める.
  \small
  \MyAlignFive
  {&SSE(\B)=\sum_i E_i=\sum_i (Y_i-\hat{Y}_i)^2
  =\sum_i (Y_i-\B x_i)^2=\sum_i (Y_i^2-2\B x_i Y_i+\B^2 x_i^2)}
    {&正規方程式は,~\frac{\partial SSE(\B)}{\partial \B}=\sum_i (-2x_i Y_i + 2\B x_i^2)=0~となる.}
  {&\Ra \qquad \sum_i(2\B x_i^2)=\sum_i(2x_i Y_i)}
  {&\Ra \qquad 2\B \sum_i x_i^2=2\sum_i x_i Y_i}
  {&\Ra \qquad \B =\frac{\sum_i x_i Y_i}{\sum_i x_i^2}\qquad(\beta の最小二乗推定量)}
  cf. $b =\frac{\sum_i x_i y_i}{\sum_i x_i^2}$\quad(\beta の最小二乗推定値)
}

\MyFrame{回帰係数の推定 (切片\red{なし}単回帰モデル)}
{
  \MyProblem
  {
    \MyFig{0.7}{problem_slope_calc_data.png}
    \MyFig{0.6}{problem_slope_calc_graph.png}
  }
}

\MyFrame{回帰係数の推定 (切片\red{なし}単回帰モデル)}
{
  \MySolution
  {
    \MyFig{0.7}{solution_slope_calc_data.png}
    \MyFig{0.6}{solution_slope_calc_graph.png}
  }
}

\MyFrame{回帰係数の推定 (切片\red{あり}単回帰モデル)}
{
  切片あり単回帰モデル($Y_i=\beta_0 + \beta_1 x_i + E_i$)の残差平方和$SSE(\B_0, \B_1)$を
  それぞれの変数に関してを偏微分し$SSE(\B_0, \B_1)$を
  最小にする切片:$\B_0$,傾き:$\B_1$を求める.
  %
  \MyAlign
  {
    &\B_1
    =\frac{\sum_i (Y_i-\bar{Y})(x_i-\bar{x})}{\sum_i (x_i-\bar{x})^2}
    =\frac{\sum_i x_i Y_i-n\bar{x}\bar{Y}}{\sum_i x_i^2-n\bar{x}^2}\\
    &\B_0=\bar{Y}-\B_1\bar{x}\\
    &ここで,\bar{x}=\sum_i x_i/n,~\bar{Y}=\sum_i Y_i/n
  }
  cf.
  \MyAlign
  {
    b_1&
    =\frac{\sum_i (y_i-\bar{y})(x_i-\bar{x})}{\sum_i (x_i-\bar{x})^2}
    =\frac{\sum_i x_i y_i-n\bar{x}\bar{y}}{\sum_i x_i^2-n\bar{x}^2}\\
    b_0&=\bar{y}-b_1\bar{x}
  }
}

\note{note\\note outside of frame}

\MyFrame{回帰係数の推定 (切片\red{あり}単回帰モデル)}
{
  \MyProblem
  {
    \MyFig{0.7}{problem_slope_calc_data.png}
    \MyFig{0.6}{problem_slope_calc_graph.png}
  }
}

\MyFrame{回帰係数の推定 (切片\red{あり}単回帰モデル)}
{
  \MySolution
  {
    \MyFig{0.7}{solution_slope_calc_data.png}
    \MyFig{0.6}{solution_slope_calc_graph2.png}
  }
}

\MyFrame{}
{
  \MyProblem
  {
    \MyFig{0.7}{quiz/toukei_kentei2.png}
  }
}

\MyFrame{}
{
  \MySolution
  {
    \MyFig{0.7}{quiz/toukei_kentei2_sol.png}
  }
}

\MyFrame{最小二乗線を適用できる条件}
{
  \MyItems
  {
    \item 誤差項の分散$\sigma$が均一であること. 
    \item 誤差間で相関が無いこと.
    \item 誤差項と説明変数の間で相関が無いこと.
  }
  これらの条件については\MyFill{残差分析}(residual analysis)で詳細に学習する.
}

%\MyFrame{最小二乗線を適用できる条件}
%{
%  \MyItems
%  {
%    \item データに線形の傾向があること.
%    \item 誤差がほぼ正規分布
%    \item 誤差変動が一定
%    \item 観測値が独立
%  }
%}
%\note
%{
%線形性. データには線形トレンドが見られる.
%非線形トレンドがあると,より高度な回帰分析を他の書物やコースで学ぶ方法が必要となる.
%ほぼ正規分布の誤差.一般的に残差の分布は正規分布に近い必要がある.
%この条件が適当ではないときには外れ値や影響のあるデータがあることが多いが, 図表 8.12 で示される残差について, 1 点が明らかに他の点を異なり直線から大きく外れている.
%一定の変動性. 最小二乗線の周りの観測点の変動は大まかに一定となる . 変動が一定ではない例はこの条件を満たさないよくあるパターンとして,x が大きくなるにつれ y の変動が大きくなっている.独立な観察値. 時系列データ,
%例えば毎日の株価などの時間と共に観察されるデータに回帰を適用するときは注意を要する.
%時系列の場合にはモデルや分析においてデータの背後にある種の構造を考慮する必要があることが多い.
%連続的な観測値は独立でない.
%}
%
%\MyFrame{【確認問題】最小二乗線の適用可能性}
%{
%  次のデータに最小二乗線を適用すべきか?
%  \MyFig{1.0}{問題_線形モデルを適用できるデータ.png}
%  \OpenIntro
%}
%\note
%{
%  本章の条件が十分でないデータの例を示す.
%  最初のパネルは線形性を満たさない.
%  第二のパネルは外れ値があり 1 点のみ直線からかなり外れた値がある.
%  第三のパネルは誤差の変動が x の値に関係している.
%  第四のパネルは時系列データが示され連続する観測値が互いに高い相関がある.
%}

\section{回帰分析(単回帰モデル) R演習}

\MyFrame{\insertsection}
{
  RStudio Cloudで,
  次のURLにあるソースコードを\red{タイプ}し回帰分析を行え.
  \alert{コピペすると記憶に定着しないためNG}
  \url{https://rpubs.com/tkdhss111/SLR}
}

\MyFrame{\insertsection}
{
  次のb0(切片),b1(傾き)の値を変えて,
  生成されるデータについて
  Rで単回帰モデルの回帰係数を推定せよ.
  \MyFig{0.5}{regression_analysis_hw_data.png}
}

\section{回帰分析(単回帰モデル) R課題}

\MyFrame{\insertsection}
{
  相関分析が終了したら,
  RStudioの右上のアイコン
  \includegraphics[width=8mm]{icon_publish.pdf}をクリックして,
  RPubsの入力画面で次の内容を入力し,
  www公開(publish)せよ.\\
  \alert{www公開されるので個人名など機微な情報は入力しないこと.}
  \begin{description}
    \item[Username] tiu学籍番号 (例)tiu22110001\\
    \item[Title] 回帰分析(単回帰モデル)\\
    \item[Description] (空白/説明を入れてもよい)\\
    \item[Slug] SLR
  \end{description}
  (職場で自分のソースコードとしてすぐに活用できるように,
    自分なりの補足説明やコメントを入れておきましょう.)
}

\end{document}
