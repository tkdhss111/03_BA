\newcommand{\Draft}{}
\newcommand{\Slide}{}
\newcommand{\PrintLecture}{1}
\newcommand{\PrintSolution}{1}
\newcommand{\MyCourse}{データサイエンスコース}
\newcommand{\MySemester}{春}
\newcommand{\MySubject}{ビジネス アナリティクス}
\newcommand{\MyClass}{第02回ーデータ収集}% フォルダ名自動挿入

%
% 科目共通定義
%

\newcommand{\OpenIntro}
{\MyRef{OpenIntro Statistics}{https://www.openintro.org/book/os}}

\newcommand{\ra}{\rightarrow}
\newcommand{\Ra}{\Rightarrow}

% Expectation E[X]
\def\E#1{E\big[#1\big]}
\def\S{\sum_{i=1}^n}

\newcommand{\B}{\hat{\beta}}
\newcommand{\SUM}{\sum_{i=1}^n}  % Summention from i=1 to n
\newcommand{\NH}{$\mathit{H}_0$} % Null hypthesis
\newcommand{\AH}{$\mathit{H}_1$} % Alternative hypothesis
\newcommand{\T}{\texorpdfstring{$t$}{}}% Student's t
\newcommand{\overtext}[3][1.5]{
  \mathrel{\overset{#2}{\scalebox{#1}[1]{$#3$}}}
}
\newcommand{\iid}{\overtext[2]{iid}{\sim}}
\newcommand{\convdist}{\overtext[2]{d}{\rightarrow}}
\newcommand{\convprob}{\overtext[2]{p}{\rightarrow}}
\newcommand{\as}[2]{\quad \text{as}\quad #1 \rightarrow #2}

\input{../../tex/hss_lualatex.tex}
\input{../../tex/hss_hyperref.tex}
\input{../../tex/hss_beamer.tex}

\setbeameroption{hide notes}
%\setbeameroption{show notes}
%\setbeameroption{show only notes}
%\setbeameroption{show notes on second screen=right}

\begin{document}

\maketitle

\MyFrame{}{\tableofcontents}

\section{section}

\MyFrame{}
{
  分析目的に合ったデータは何か検討する.
}

\MyFrame{データの有用性}
{
  test
}

%ボツ:RFIDの話がBA向けではない。
%\MyFrame{データ収集技術とウェアラブルデバイス}
%{
%  【総務省ICTスキル総合習得プログラム】\\
%  \url{https://www.soumu.go.jp/ict_skill/making.html}\\
%  1-2:データ収集技術とウェアラブルデバイス(YouTube動画; 11:08)\\
%  \vfill
%  \MyRef
%  {【総務省】ICTスキル総合習得プログラム 編集加工}
%  {https://www.soumu.go.jp/ict_skill}
%}

\MyFrame{APIによるデータ収集と利活用}
{
  【総務省ICTスキル総合習得プログラム】\\
  \url{https://www.soumu.go.jp/ict_skill/making.html}\\
  1-5 APIによるデータ収集と利活用(YouTube動画; 6:52)\\
  \vfill
  \MyRef
  {【総務省】ICTスキル総合習得プログラム 編集加工}
  {https://www.soumu.go.jp/ict_skill}
}

\end{document}
