\newcommand{\Draft}{}
\newcommand{\Slide}{}
\newcommand{\PrintLecture}{1}
\newcommand{\PrintSolution}{1}
\newcommand{\MyCourse}{データサイエンスコース}
\newcommand{\MySemester}{春}
\newcommand{\MySubject}{ビジネス アナリティクス}
\newcommand{\MyClass}{第02回ーデータ収集}% フォルダ名自動挿入

%
% 科目共通定義
%

\newcommand{\OpenIntro}
{\MyRef{OpenIntro Statistics}{https://www.openintro.org/book/os}}

\newcommand{\ra}{\rightarrow}
\newcommand{\Ra}{\Rightarrow}

% Expectation E[X]
\def\E#1{E\big[#1\big]}
\def\S{\sum_{i=1}^n}

\newcommand{\B}{\hat{\beta}}
\newcommand{\SUM}{\sum_{i=1}^n}  % Summention from i=1 to n
\newcommand{\NH}{$\mathit{H}_0$} % Null hypthesis
\newcommand{\AH}{$\mathit{H}_1$} % Alternative hypothesis
\newcommand{\T}{\texorpdfstring{$t$}{}}% Student's t
\newcommand{\overtext}[3][1.5]{
  \mathrel{\overset{#2}{\scalebox{#1}[1]{$#3$}}}
}
\newcommand{\iid}{\overtext[2]{iid}{\sim}}
\newcommand{\convdist}{\overtext[2]{d}{\rightarrow}}
\newcommand{\convprob}{\overtext[2]{p}{\rightarrow}}
\newcommand{\as}[2]{\quad \text{as}\quad #1 \rightarrow #2}

\input{../../tex/hss_lualatex.tex}
\input{../../tex/hss_hyperref.tex}
\input{../../tex/hss_beamer.tex}

\setbeameroption{hide notes}
%\setbeameroption{show notes}
%\setbeameroption{show only notes}
%\setbeameroption{show notes on second screen=right}

\begin{document}

\maketitle

\MyFrame{}{\tableofcontents}

\section{データの型}

\MyFrame{\insertsection}
{
  \begin{description}[浮動小数点型]
    \item [NULL型] NULL値(値が無いこと) (null) 
    \item [論理型] 真[true]/偽[false] (logical, boolean) 
    \item [日時型] 日付や時刻(datetime, date, time) 
    \item [バイナリ型] バイナリ形式データ(blob) (e.g., 画像) 
    \item [文字列型] 文字(text, character, string) 
    \item [整数型] 整数(integer)
    \item [浮動小数点型] 実数
      \begin{itemize}
        \item 単精度(real, float, single)
        \item 倍精度(real8, double, numeric)
      \end{itemize}
  \end{description}
  SQLiteのREALは倍精度浮動小数点型
}

\MyFrame{浮動小数点数(floating point number)とは}
{
  数値を整数部と指数部で表わしたもの.
  計算機ではこの表現で数値を扱う.
  \[(固定小数点数)~1.2345=\underbracket{12345}_{仮数部}\times {\underbracket{10}_{基数}}^{\overbracket{-4}^{指数部}}~(浮動小数点数)\]
  \MyFig{0.5}{fixed-floating.png}
  \MyCap{固定小数点数と浮動小数点数の違い}
  \MyRef
  {固定小数点数と浮動小数点数の違いを調べよう!}
  {https://itmanabi.com/fixed-floating}
}

\section{データファイル形式}% ToDo add more contents here

\MyFrame{\insertsection}
{
  \begin{description}[浮動小数点型]
    \item [csv] カンマ区切り値ファイル(Comma separated values)
    \item [txt] テキストファイル
    \item [json] \href{https://camp.trainocate.co.jp/magazine/whats-json}{JSONファイル}
    \item [xml] XMLファイル
    \item [xls] Excelファイル形式
      \begin{itemize}
        \item [xlsm]マクロ付きExcel
        \item [xlsx]XML形式Excel
      \end{itemize}
  \end{description}
}

\section{データクレンジング}

\MyFrame{}
{
  \MyFig{1.0}{品質の悪いデータによる社会的損失}

  \MyRef
  {【総務省 ICTスキル総合習得教材】3-1}
  {https://www.soumu.go.jp/ict_skill/pdf/ict_skill_3_1.pdf}
}

\MyFrame{}
{
  \MyFig{1.0}{データクレンジングツール}

  \MyRef
  {【総務省 ICTスキル総合習得教材】3-1}
  {https://www.soumu.go.jp/ict_skill/pdf/ict_skill_3_1.pdf}
}

\MyFrame{}
{
  \MyFig{1.0}{データクレンジングの負担}

  \MyRef
  {【総務省 ICTスキル総合習得教材】3-1}
  {https://www.soumu.go.jp/ict_skill/pdf/ict_skill_3_1.pdf}
}

\MyFrame{}
{
  \MyFig{1.0}{国内企業におけるデータ分析の実態}

  \MyRef
  {【総務省 ICTスキル総合習得教材】3-1}
  {https://www.soumu.go.jp/ict_skill/pdf/ict_skill_3_1.pdf}
}

\section{APIを利用したデータ収集}

\MyFrame{\insertsection}
{
  次の出典に示す教材の【参考2】を参照すること.\\
  \MyRef
  {【総務省 ICTスキル総合習得教材】4-3}
  {https://www.soumu.go.jp/ict_skill/pdf/ict_skill_4_3.pdf}

  \href{https://www.e-stat.go.jp}{【政府統計の総合窓口(e-Stat)】}   
}

\section{Web scrapingによるデータ収集 R演習}

\MyFrame{\insertsection}
{
  RStudio Cloudで,
  次のURLにあるソースコードを\red{タイプ}し気象庁から気象データを取得せよ.
  \alert{コピペすると記憶に定着しないためNG}
  \url{https://rpubs.com/tkdhss111/data_acquisition}
}

\end{document}
