\newcommand{\Draft}{}
\newcommand{\Slide}{}
\newcommand{\PrintLecture}{1}
\newcommand{\PrintSolution}{1}
\newcommand{\MyCourse}{データサイエンスコース}
\newcommand{\MySemester}{春}
\newcommand{\MySubject}{ビジネス アナリティクス}
\newcommand{\MyClass}{第02回ーデータ収集}% フォルダ名自動挿入

%
% 科目共通定義
%

\newcommand{\OpenIntro}
{\MyRef{OpenIntro Statistics}{https://www.openintro.org/book/os}}

\newcommand{\ra}{\rightarrow}
\newcommand{\Ra}{\Rightarrow}

% Expectation E[X]
\def\E#1{E\big[#1\big]}
\def\S{\sum_{i=1}^n}

\newcommand{\B}{\hat{\beta}}
\newcommand{\SUM}{\sum_{i=1}^n}  % Summention from i=1 to n
\newcommand{\NH}{$\mathit{H}_0$} % Null hypthesis
\newcommand{\AH}{$\mathit{H}_1$} % Alternative hypothesis
\newcommand{\T}{\texorpdfstring{$t$}{}}% Student's t
\newcommand{\overtext}[3][1.5]{
  \mathrel{\overset{#2}{\scalebox{#1}[1]{$#3$}}}
}
\newcommand{\iid}{\overtext[2]{iid}{\sim}}
\newcommand{\convdist}{\overtext[2]{d}{\rightarrow}}
\newcommand{\convprob}{\overtext[2]{p}{\rightarrow}}
\newcommand{\as}[2]{\quad \text{as}\quad #1 \rightarrow #2}

\input{../../tex/hss_lualatex.tex}
\input{../../tex/hss_hyperref.tex}
\input{../../tex/hss_beamer.tex}

\setbeameroption{hide notes}
%\setbeameroption{show notes}
%\setbeameroption{show only notes}
%\setbeameroption{show notes on second screen=right}

\begin{document}

\maketitle

\MyFrame{}{\tableofcontents}

%\MyFrame{}
%{
%  \MyDefinition{パーセンタイル}%ToDo どこで教える.確率・統計?
%  {
%  }
%}

\section{四分位数}

\MyFrame{}
{
  \MyDefinition{四分位数}
  {
    データの値を小さい順に並べたとき,4等分する位置の値を
    \MyFill{\ruby{四分位数}{しぶんいすう}}(quartile points)という.
    小さい方から順に,
    \MyItems
    {
      \item 第1四分位数($Q_1$)
      \item 第2四分位数($Q_2$)・・・中央値(median)
      \item 第3四分位数($Q_3$)
    }
    \MyFig{0.9}{quantile_order.png}
  }
  \MyRef{【WAKARA】四分位数と箱ひげ図}
  {https://wakara.co.jp/mathlog/20210927}
}

\MyFrame{\insertsection}
{
  \MyFig{1.0}{quantile.png}
  \MyRef{【WAKARA】四分位数と箱ひげ図}
  {https://wakara.co.jp/mathlog/20210927}
}

\MyFrame{}
{
  \MyDefinition{箱ひげ図}
  {
    最小値,$Q_1$,$Q_2$,$Q_3$,最大値の5つの数値で
    分布を要約(五数要約)したグラフのことを
    \MyFill{箱ひげ図}(box plot/box-and-whisker plot)という.
    \MyFig{0.7}{quantile_boxplot.png}
  }
  ただし,ここでの最小値,最大値は,箱の端から\alert{IQRの1.5倍の範囲内}での値とする.IQR$=Q_3-Q_1$はデータの半分が含まれる範囲.\\
  \MyRef{【数学の景色】箱ひげ図とは~わかりやすく図解}
  {https://mathlandscape.com/box-plot}
}

\MyFrame{ヒストグラムと箱ひげ図の対応}
{
  \MyFig{1.0}{histogram_boxplot.png}
  \MyRef{【数学の景色】箱ひげ図とは~わかりやすく図解}
  {https://mathlandscape.com/box-plot}
}

\MyFrame{ヒストグラムと箱ひげ図の対応}
{
  \MyFig{0.7}{histogram_boxplot1.png}
  \MyRef{【高校数学.net】}
  {https://高校数学.net/hakohige-hisutoguramu}
}

\MyFrame{ヒストグラムと箱ひげ図の対応}
{
  \MyFig{0.7}{histogram_boxplot2.png}
  \MyRef{【高校数学.net】}
  {https://高校数学.net/hakohige-hisutoguramu}
}

\MyFrame{ヒストグラムと箱ひげ図の対応}
{
  \MyFig{0.7}{histogram_boxplot3.png}
  \MyRef{【高校数学.net】}
  {https://高校数学.net/hakohige-hisutoguramu}
}

\MyFrame{ヒストグラムと箱ひげ図の対応}
{
  \MyFig{0.7}{histogram_boxplot4.png}
  \MyRef{【高校数学.net】}
  {https://高校数学.net/hakohige-hisutoguramu}
}

\MyFrame{ヒストグラムと箱ひげ図の対応}
{
  \MyFig{0.7}{histogram_boxplot5.png}
  \MyRef{【高校数学.net】}
  {https://高校数学.net/hakohige-hisutoguramu}
}

\MyFrame{箱ひげ図の長所}
{
  ヒストグラムでは難しい複数の分布の同時比較ができる.
  \MyFig{0.35}{boxplot_merit.png}
}

\MyFrame{箱ひげ図の短所}
{
  分布の詳細な形状は分からない.
  全く同じ箱ひげ図であっても分布は異なる場合がある.
  (cf バイオリン図)
  \MyFig{0.7}{boxplot_pitfall.pdf}
  \MyCap{異なるヒストグラム→同じ箱ひげ図}
}

\MyFrame{バイオリン図(推奨)}
{
  箱ひげ図の周りに分布形状を表す線を引いたグラフ.
  箱ひげ図の上位互換だが市民権を得ていないため,
  他者へは多少説明を要す.\\[3mm]
  \begin{minipage}{0.45\textwidth}
    \MyFig{1.0}{histogram-vioplot.png}
  \end{minipage}
  \begin{minipage}{0.45\textwidth}
    \MyFig{1.0}{violinplot_graph.png}
  \end{minipage}
  \MyCap{バイオリン図(右)}
  \MyRef{【R CODER】Violin plot in R}
  {https://r-coder.com/violin-plot-r}
}

\section{箱ひげ図 問題演習}

\MyFrame{\insertsection}
{
  \MyProblem
  {
    \MyFig{1.0}{quiz/problem44.png}
  }
  \vspace{40mm}
}

\MyFrame{\insertsection}
{
  \MySolution
  {
    \MyFig{0.9}{quiz/solution44.png}
  }
}

\MyFrame{\insertsection}
{
  \MyProblem
  {
    \MyFig{1.0}{quiz/problem41.png}
  }
  \vspace{40mm}
}

\MyFrame{\insertsection}
{
  \MySolution
  {
    \MyFig{0.9}{quiz/solution41.png}
  }
}

\MyFrame{\insertsection}
{
  \MyProblem
  {
    \MyFig{1.0}{quiz/problem42.png}
  }
  \vspace{40mm}
}

\MyFrame{\insertsection}
{
  \MySolution
  {
    \MyFig{0.8}{quiz/solution42.png}
  }
}

\MyFrame{\insertsection}
{
  \MyProblem
  {
    \MyFig{1.0}{quiz/problem45.png}
  }
  \vspace{40mm}
}

\MyFrame{\insertsection}
{
  \MySolution
  {
    \MyFig{1.0}{quiz/solution45.png}
  }
}

\MyFrame{\insertsection}
{
  \MyProblem
  {
    \MyFig{1.0}{quiz/problem54.png}
  }
  \vspace{40mm}
}

\MyFrame{\insertsection}
{
  \MySolution
  {
    \MyFig{1.0}{quiz/solution54.png}
  }
}

\MyFrame{\insertsection}
{
  \MyProblem
  {
    \MyFig{1.0}{quiz/problem14.png}
  }
  \vspace{40mm}
}

\MyFrame{\insertsection}
{
  \MySolution
  {
    \MyFig{1.0}{quiz/solution14.png}
  }
}

\section{箱ひげ図 R演習}

\MyFrame{\insertsection}
{
  RStudio Cloudで,
  次のURLにあるソースコードを\red{タイプ}し箱ひげ図を作成せよ.
  \alert{コピペすると記憶に定着しないためNG}
  \url{https://rpubs.com/tkdhss111/boxplot}
}

\MyFrame{}
{
  \MyFig{0.9}{fig/boxplot_data.png}

}

\MyFrame{R カラーパレット}
{
  \MyFig{0.8}{R_color_palette.png}
  \MyFig{0.7}{RGB.png}
  \MyRef{【ウィキペディア】RGB}{https://ja.wikipedia.org/wiki/RGB}
}

\MyFrame{}
{
  \MyFig{1.0}{fig/boxplot_src.png}
}

\MyFrame{}
{
  \MyFig{1.0}{fig/boxplot_graph.png}
}

\MyFrame{}
{
  \MyFig{0.8}{fig/violinplot_src.png}
}

\MyFrame{}
{
  \MyFig{1.0}{violinplot_graph.png}

}

\section{箱ひげ図 R課題}

\MyFrame{箱ひげ図 R課題}
{
  \MyProblem
  {
    Rで箱ひげ図とバイオリン図を作成せよ.
    \MyFig{0.8}{quiz/problem56.png}
    \MyFig{0.5}{quiz/problem56_hw_data.png}
  }
}

\MyFrame{\insertsection}
{
  \MySolution
  {
    \MyFig{0.7}{quiz/solution56.png}
  }
}

\MyFrame{\insertsection}
{
  図が完成したら,
  RStudioの右上のアイコン
  \includegraphics[width=8mm]{icon_publish.pdf}をクリックして,
  RPubsの入力画面で次の内容を入力し,
  www公開(publish)せよ.\\
  \alert{www公開されるので個人名など機微な情報は入力しないこと.}
  \begin{description}
    \item[Username] tiu学籍番号 (例)tiu22110001\\
    \item[Title] 箱ひげ図\\
    \item[Description] (空白/説明を入れてもよい)\\
    \item[Slug] boxplot
  \end{description}
  (職場で自分のソースコードとしてすぐに活用できるように,
    自分なりの補足説明やコメントを入れておきましょう.)\\
    ここでの方法に追加して,
    他のRのパッケージを使って作図したら評点に加えます.
}

\MyFrame{インタラクティブなRパッケージ: \textbf{plotly}}
{
  インタラクティブ(interactive)なグラフを作成できるパッケージ.\\
  (構文がパイプ方式でやや特殊)
  \MyFig{0.5}{violinplot_plotly.png}
  \MyRef{【plotly】Violin Plots in R}{https://plotly.com/r/violin}
}

\end{document}
