\newcommand{\Release}{}
\newcommand{\Slide}{}
\newcommand{\PrintLecture}{1}
\newcommand{\PrintSolution}{1}
\newcommand{\MyCourse}{データサイエンスコース}
\newcommand{\MySemester}{春}
\newcommand{\MySubject}{ビジネス アナリティクス}
\newcommand{\MyClass}{第02回ーデータ収集}% フォルダ名自動挿入

%
% 科目共通定義
%

\newcommand{\OpenIntro}
{\MyRef{OpenIntro Statistics}{https://www.openintro.org/book/os}}

\newcommand{\ra}{\rightarrow}
\newcommand{\Ra}{\Rightarrow}

% Expectation E[X]
\def\E#1{E\big[#1\big]}
\def\S{\sum_{i=1}^n}

\newcommand{\B}{\hat{\beta}}
\newcommand{\SUM}{\sum_{i=1}^n}  % Summention from i=1 to n
\newcommand{\NH}{$\mathit{H}_0$} % Null hypthesis
\newcommand{\AH}{$\mathit{H}_1$} % Alternative hypothesis
\newcommand{\T}{\texorpdfstring{$t$}{}}% Student's t
\newcommand{\overtext}[3][1.5]{
  \mathrel{\overset{#2}{\scalebox{#1}[1]{$#3$}}}
}
\newcommand{\iid}{\overtext[2]{iid}{\sim}}
\newcommand{\convdist}{\overtext[2]{d}{\rightarrow}}
\newcommand{\convprob}{\overtext[2]{p}{\rightarrow}}
\newcommand{\as}[2]{\quad \text{as}\quad #1 \rightarrow #2}

\input{../../tex/hss_lualatex.tex}
\input{../../tex/hss_moodle.tex}
\input{../../tex/hss_beamer.tex}

\begin{document}

\maketitle

\section{小テスト(\MyClass)}

\begin{quiz}{\MyClass}

\QuizShortAnswer
{
  次の箱ひげ図で,真ん中の太線で示される値のことを何というか?
  \MyFig{0.4}{quiz_boxplot_outlier.png}
}
{
  中央値(median)
}
{30}
{中央値}
{median}
{第2四分位数}
{Q2}

\QuizShortAnswer
{
  次の箱ひげ図で,「○」で示される値のことを何というか?
  \MyFig{0.4}{quiz_boxplot_outlier.png}
}
{
  外れ値(outlier)
}
{30}
{外れ値}
{outlier}
{}
{}

\QuizMultipleChoices
{
  箱ひげ図で箱の範囲に含まれるデータは
  全体のデータの何%か?
}
{
  全データの半分
}
{30}
{25%}
{*50%}
{75%}
{不明}

\QuizShortAnswer
{
  次の図を何というか?
  \MyFig{0.7}{quiz_violin.png}
}
{
  バイオリン図(violin plot)
}
{10}
{バイオリン*}
{violin*}
{}
{}

\end{quiz}

\end{document}
