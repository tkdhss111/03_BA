\newcommand{\Release}{}
\newcommand{\Slide}{}
\newcommand{\PrintLecture}{1}
\newcommand{\PrintSolution}{0}
\newcommand{\MyCourse}{データサイエンスコース}
\newcommand{\MySemester}{春}
\newcommand{\MySubject}{ビジネス アナリティクス}
\newcommand{\MyClass}{第17回ー分類}% フォルダ名自動挿入

%
% 科目共通定義
%

\newcommand{\OpenIntro}
{\MyRef{OpenIntro Statistics}{https://www.openintro.org/book/os}}

\newcommand{\R}{\textbf{R}}
\newcommand{\RStudio}{\textbf{RStudio}}
\newcommand{\Excel}{\textbf{Excel}}
\newcommand{\cs}[1]{\textcolor{blue}{\texttt{#1}}} % Console prompt >

\newcommand{\ra}{\rightarrow}
\newcommand{\Ra}{\Rightarrow}

% Expectation E[X]
\def\E#1{E\big[#1\big]}
\def\S{\sum_{i=1}^n}

\newcommand{\B}{\hat{\beta}}
\newcommand{\SUM}{\sum_{i=1}^n}  % Summention from i=1 to n
\newcommand{\NH}{$\mathit{H}_0$} % Null hypthesis
\newcommand{\AH}{$\mathit{H}_1$} % Alternative hypothesis
\newcommand{\T}{\texorpdfstring{$t$}{}}% Student's t
\newcommand{\overtext}[3][1.5]{
  \mathrel{\overset{#2}{\scalebox{#1}[1]{$#3$}}}
}
\newcommand{\iid}{\overtext[2]{iid}{\sim}}
\newcommand{\convdist}{\overtext[2]{d}{\rightarrow}}
\newcommand{\convprob}{\overtext[2]{p}{\rightarrow}}
\newcommand{\as}[2]{\quad \text{as}\quad #1 \rightarrow #2}

\input{../../tex/hss_lualatex.tex}
\input{../../tex/hss_hyperref.tex}
\input{../../tex/hss_beamer.tex}

\setbeameroption{hide notes}
%\setbeameroption{show notes}
%\setbeameroption{show only notes}
%\setbeameroption{show notes on second screen=right}

\newcommand{\R}{$\mathbf{R}$}
\newcommand{\RS}{$\mathbf{RStudio}$}
\newcommand{\Excel}{$\mathbf{Excel}$}

\begin{document}

\maketitle

\MyFrame{}{\tableofcontents}

\section{はじめに}

\MyFrame{\insertsection}
{
  統計解析は,10年ほど前までは,CやFortranなど,
  取扱いに専門的知識を要する
  プログラミング言語を用いて行われていました.
  これを,高度なプログラミング知識がなくても
  誰でも利用できる形にしたものが,統計解析環境\R です.
  Pythonとともにデータサイエンスの世界では標準の
  ソフトウェアツールとなっています.\\
  \R を習得すれば,統計解析から業務効率化ツール
  の作成までオールマイティーに,そのスキルを活用できます.
  データ分析に携わる方は,
  様々な市販のソフトウェアに手を出さなくても,
  これ一本で十分です.\\
  手を動かして自分でやってみることが\R 習得の近道です.
  そのため,各スライドには,
  必ず演習を載せてあります.
}

\section{統計解析環境 \R}

\MyFrame{\insertsection}
{
  \R とは,AT&Tベル研究所が開発した統計解析用の
  プログラミング言語(S言語)を参考にして作られた
  フリーのオープンソース言語(R言語)
  を使用できる統計解析環境.\\[3mm]

  \MyCols
  {
    \MyFig{0.5}{logo-r.eps}
    \MyCap{The R environment}
  }
  {
    \MyFig{0.5}{r-open.pdf}
    \MyCap{Microsoft R Open}
  }
  \R により,現代統計学をほぼ網羅する広範な統計解析や
  出版物品質のグラフ描画が容易に可能となる.
  Microsoft社により開発・保守されている高速版の\R
  (Microsoft R Open) も存在する.
}

\MyFrame{\R パッケージ}
{
  \R だけでも基本的な統計解析は可能だが,
  ユーザーの利用目的に応じて開発された\R パッケージと呼ばれる
  統計解析ライブラリをインストールすることで機能を拡張できる.\\
  
  \R パッケージは,C/C++,Fortran,R言語で記述されており,
  当初は,欧米大学の統計学科の教員らが中心となり
  開発・保守を行っていたが,
  近年は民間を含む様々な分野で広く開発が進められている.
  すでに,1,000を超える\R パッケージがウェブ上で公開されている.
}

\MyFrame{CRAN}
{
  CRAN(包括的R保存庫網)とは,\R の本体やパッケージ,
  マニュアル類が無償公開されているウェブサイト.
  ユーザーは,最寄りのミラーサイトからソフトウェア
  をダウンロードする.\\[3mm]
  %\R 関連のソフトウェアをダウンロードする際には,
  %次の国内のサイトを利用する.

  %$\rightarrow$ 統計数理研究所(\url{https://cran.ism.ac.jp})
  \MyFig{0.6}{cran.png}
}

\MyFrame{RjpWiki}
{
  日本語での\R の情報源としては,
筑波大学の岡田氏のウェブサイト\blue{RjpWiki}が有名.
  多くの有用な情報が掲載されており,質問もできる.\\

  \MyFig{1.0}{rjpwiki}
  \MyRef{RjpWiki}{http://www.okadajp.org/RWiki}
}

\MyFrame{統合開発環境 \RS}
{
  \RS とは,\R 用の統合開発環境(IDE)で,
  ソースコードの編集,実行,ヘルプの表示,
  パッケージの作成など,
  プログラミングに必要な様々な便利な機能を持つ.
  Pythonの開発もできる.
  オープンソース版の\RS をrstudio.comからダウンロードできる. 
  \MyFig{0.5}{rstudio}
  \MyRef{rstudio.com}{https://rstudio.com}
}

\section{基本操作}

\MyFrame{\insertsection~スカラの作成}
{
  \MyDefinition{手順}
  {
    オブジェクト名の後に,代入(付置)記号「\texttt{<-}」と値を入力する.
  }
  ・「\texttt{<-}」の代わりに「=」も使用できる (若干意味が異なる).\\
  ・オブジェクト名は,大文字と小文字は区別される.
  \MyCols
  {
    \MyConsole
    {
      \MyCmd{x <- 1}\\
      \MyCmd{kw.pv <- 3.1}
    }
  }
  {
    \MyConsole
    {
      \MyCmd{ls()} \MyCmt{オブジェクト名表示}\\
      \MyCmd{rm(list=ls())} \MyCmt{全消去}
    }
  }

  \MyExercise
  {
    「Alt~+~-」で代入記号(\texttt{<-})を20回入力せよ.\\
    オブジェクト名を表示,オブジェクトを削除せよ.
  }
}
\note
{
  ・\texttt{ls()}で作成したオブジェクトの名前を表示できる.\\
  ・\texttt{rm('x')}でオブジェクトxを削除できる. 
  \texttt{rm(list=ls())}で全消去.\\
  \R ではオブジェクト名称の大文字と小文字は区別される.
  漢字名称も使用可能だが,通常はローマ字小文字で,
  判別可能な略語を用い,ofの意味で「.」や「\_」
  を利用し単語を結合すると分かりやすくなる.
  【例】\texttt{lat.jp}(日本の緯度),\texttt{kw.pv}(PV発電量)
}

\MyFrame{\insertsection~ベクトルの作成}
{
  \MyColsThree
  {
    \MyDefinition{手順(方法1)}
    {結合関数「\texttt{c}」を用いて作成}
  }
  {
    \MyDefinition{手順(方法2)}
    {等差数列作成記号「\texttt{:}」を用いて作成}
  }
  {
    \MyDefinition{手順(方法3)}
    {等差数列作成関数「\texttt{seq}」を用いて作成}
  }
  「\texttt{rep}」関数で同一値ベクトル作成も可能
  \texttt{rep(NA, 3)} $\ra$ \texttt{NA NA NA}\\
  「?関数名」をコンソールに入力するとヘルプが表示される.

  \MyColsThree
  {
    \MyConsole
    {
      \MyCmd{v <- c(1,6,3)}\\ \relax [1] 1 6 3
    }
  }
  {
    \MyConsole
    {
      \MyCmd{v <- 1:3}\\ \relax[1] 1 2 3
    }
  }
  {
    \MyConsole
    {
      \MyCmd{v <- seq(1,6,2)}\\ \relax [1] 1 3 5
    }
  }
}

\MyFrame{\insertsection~ベクトルの作成}
{
  \MyExercise
  {
    次のベクトルを作成せよ.
    $(3, 2, 1),(3, 6, 9),(4, 2, 0),(1.5, 2.5, 3.5),(1, 2, 3, 1, 2, 3)$
  }
}

\MyFrame{\insertsection~行列の作成}
{
  \MyDefinition{手順}
  {
    行列作成関数「\texttt{matrix}」を用いて作成.
    行数:\texttt{nrow},列数:\texttt{ncol}
  }
  \MyConsole
  {
    \MyCmd{m <- matrix(1:4, nrow = 2, ncol = 2)}\\
    \hfill (オプション byrow = T で行毎に値代入)\\
    \MyCmd{m}\\
    % ToDo: verbatim did not work even with fragile option
    \relax
    \hspace{3mm}[,1] [,2]\\ \relax
        [1,]    1    3\\ \relax
        [2,]    2    4
  }
}

\MyFrame{\insertsection~行列の作成}
{
  \MyExercise
  {
    次の行列を作成せよ.
    \vspace{-9mm}
    \[
      \hspace{40mm}
      \begin{bmatrix}
        -2 & 0 & 2\\
         4 & 6 & 8 
      \end{bmatrix}
      , 
      \begin{bmatrix}
        NA & NA\\
        NA & NA 
      \end{bmatrix}
    \]
  }
}

\MyFrame{\insertsection~オブジェクトの画面表示}
{
  \MyDefinition{手順}
  {
    コンソールに表示させたいオブジェクト名を入力する.
    または,ソースコード画面でオブジェクトを選択して
    Ctrl+Enterを押す.
  }
  \MyColsThree
  {
    \MyConsole
    {
      \MyCmd{v[3]}\\ \relax [1] 3
    }
  }
  {
    \MyConsole
    {
      \MyCmd{m[2, ]}\\ \relax [1] 2 4
    }
  }
  {
    \MyConsole
    {
      \MyCmd{(x <- c(1, 2))}\\ \relax [1] 1 2
    }
  }
  【オブジェクト要素の\R 表記】\\
   ~~ベクトルvの要素$i$: v[i]\\
   ~~行列mの$i$行: \texttt{m[i, ]},
   $j$列: \texttt{m[, j]},
   要素$(i, j)$: \texttt{m[i, j]}
}

\MyFrame{\insertsection~オブジェクトの画面表示}
{
  \MyExercise
  {
    スカラ,ベクトル,行列の値を表示せよ.
  }
}

\MyFrame{\insertsection~データフレームの作成}
{
  \MyDefinition{手順}
  {
    テーブル作成関数「data.frame」を用いて作成する.
  }
  \MyConsole
  {
    \MyCmd{d <- data.frame(name = c('panda', 'lion'),\\
           \hfill age  = c(5, 7), male = c(T, F))}

    \MyCmd{d}\\
       ~~~name age  male\\
     1 panda 5 TRUE\\
     2 ~~~lion 7 FALSE\\
  }
}

\MyFrame{\insertsection~データフレームの作成}
{
  \MyExercise
  {
    漢字,数値,論理値のカラムを持つ,
    データフレームを作成せよ(内容自由).
  }
}

\MyFrame{\insertsection~データフレームの操作1}
{
  \MyDefinition{手順}
  {
    アクセスしたいコラム(列)やレコード(行)の
    インデック番号を入力する.
    負の番号を入れると,そのコラムが除かれる.
  }
  \MyColsThree
  {
    \MyConsole
    {
      \MyCmd{d[, 1]}\\ \relax
      [1] "panda" "lion" 
    }
  }
  {
    \MyConsole
    {
      \MyCmd{d[, c(1, 3)]}\\ \relax
      ~~name  male\\
      1 panda  TRUE\\
      2 ~~~lion FALSE
    }
  }
  {
    \MyConsole
    {
      \MyCmd{d[, -1]}\\ \relax
      ~~age  male\\
      1   5  TRUE\\
      2   7 FALSE
    }
  }
}

\MyFrame{\insertsection~データフレームの操作1}
{
  \MyExercise
  {
    レコード(行)にもインデックス番号でアクセスし,
    値の表示や代入を行え.
  }
}

\MyFrame{\insertsection~データフレームの操作2}
{
  \MyDefinition{手順}
  {
    オブジェクト名のあとにアクセスしたいコラム(列)名
    を$で結びつける.
    または,コラム(列)名をリテラルで囲み記入する.\\
    データフレームdのカラム: d\$カラム名 or d[, 'カラム名']
  }
  \MyCols
  {
    \MyConsole
    {
      \MyCmd{ d\$age }\\ \relax
      [1] 5 7
    }
  }
  {
    \MyConsole
    {
      \MyCmd{ d[, c('name', 'age')] }\\
      ~~~name age\\
        1 panda   5\\
        2 ~~~lion   7
    }
  }
}

\MyFrame{\insertsection~データフレームの操作2}
{
  \MyExercise
  {
    レコード(行)にもインデックス番号でアクセスし,
    値の表示や代入を行え.
    カラム名,レコード名は
    \texttt{rownames(d)}, \texttt{colnames(d)}でアクセス可能.
  }
}

\MyFrame{\insertsection~データフレームの操作3}
{
  \MyDefinition{手順}
  {
    アクセスしたいコラム(列)やレコード(行)に論理値を入力する.
    T(TRUE)の論理値箇所のデータが抽出される.
  }
  \MyCols
  {
    \MyConsole
    {
      \MyCmd{ d[, c(T, F, T)] }\\
          ~~~name  male\\
        1 panda  TRUE\\
        2 ~~~lion FALSE
    }
  }
  {
    \MyConsole
    {
      \MyCmd{ d[d\$age > 6, ] }\\
       ~~name age  male\\
       2 ~~lion ~~7 FALSE
    }
  }
}

\MyFrame{\insertsection~データフレームの操作3}
{
  \MyExercise
  {
    レコード(行)に論理値ベクトルでアクセスし,
    値の表示や代入を行え.
  }
}

\section{演算}

\MyFrame{\insertsection~四則}
{
  \MyDefinition{手順}
  {
    和「\texttt{+}」,
    減「\texttt{-}」,
    積「\texttt{*}」,
    除「\texttt{/}」,
    乗「\texttt{\ \^} 」,
    剰余「\texttt{\%\%}」,
    剰商「\texttt{\%/\%}」の算法記号を使って演算する.
    要素ごとの演算となる.
  }
  この他,行列演算用の積「\texttt{\%*\%}」,
  転置「\texttt{t()}」,
  逆行列「\texttt{solve()}」などもある.
  \MyColsThree
  {
    \MyConsole
    {
      \MyCmd{x <- 1:3; y <- 1:3}\\
      \MyCmd{x + y}\\ \relax
      [1] 2 4 6
    }
  }
  {
    \MyConsole
    {
      \MyCmd{x <- 1:3; y <- 1:3}\\
      \MyCmd{x * y}\\ \relax
      [1] 1 4 9
    }
  }
  {
    \MyConsole
    {
      \MyCmd{x <- 9; y <- 2}\\
      \MyCmd{x \%/\% y}\\ \relax
      [1] 4
    }
  }
}

\MyFrame{\insertsection~四則}
{
  \MyExercise
  {
    既出のすべての演算記号を使って計算せよ(内容自由).
  }
}

\MyFrame{\insertsection~組込関数}
{
  \MyDefinition{手順}
  {
    平均値「\texttt{mean}」,
    中央値「\texttt{median}」,
    最大「\texttt{max}」,
    最小「\texttt{min}」\\
    範囲「\texttt{range}」,
    平方根「\texttt{sqrt}」,
    絶対値「\texttt{abs}」,
    丸め「\texttt{round}」など
  }
  \MyColsThree
  {
    \MyConsole
    {
      \MyCmd{x <- 1:3}\\
      \MyCmd{mean(x)}\\ \relax
      [1] 2
    }
  }
  {
    \MyConsole
    {
      \MyCmd{x <- 1:3}\\
      \MyCmd{range(x)}\\ \relax
      [1] 1 3
    }
  }
  {
    \MyConsole
    {
      \MyCmd{x <- 3.14}\\
      \MyCmd{round(x, 1)}\\ \relax
      [1] 3.1 
    }
  }
}

\MyFrame{\insertsection~組込関数}
{
  \MyExercise
  {
    既出のすべての組み込み関数を使って,計算せよ(内容自由).
    また,その他の関数,定数(\texttt{pi})は,
    どのようなものがあるかウェブで検索せよ.
  }
}

\MyFrame{\insertsection~自作関数1}
{
  \MyDefinition{手順}
  {
    関数オブジェクト名 \texttt{<- function (引数1,引数2,...)} 関数式\\
    の形式で関数を作成する.引数は値渡しとなる.
  }
  \MyConsole
  {
    \MyCmd{get.mbe <- function(yhat, y) mean(yhat - y)}\\
    \MyCmd{mbe <- get.mbe(yhat \red{=} 1:3, y \red{=} 4:6)}\\\relax
    [1] -3
  }
}

\MyFrame{\insertsection~自作関数1}
{
  \MyExercise
  {
    RMSE(平均2乗誤差平方)を求める関数を作成せよ.
    \vspace{-2mm}
    \[
      \mathrm{RMSE}=\sqrt{\frac{1}{N}\sum_{i=1}^{N}(\hat{y}_i-y_i)^2}
    \]
    cf. 平均:~\texttt{mean()},平方根:~\texttt{sqrt()}
  }
}

\MyFrame{\insertsection~自作関数2}
{
  \MyDefinition{手順}
  {
    複数行の関数を作成するときは,
    関数のスコープを示す{}(中括弧)や
    変数名を指定して出力する\texttt{return}関数を用いる.
  }
  \MyCols
  {
    \MyConsole
    {
      \MyCmd
      {
        f <- function(x) \{\\
          ~~~~~y~<- 1 + x + x \^ \ 2\\
          ~~~~~z~<- log(y)\\
          ~~~~~return(z)\\
          ~~~\}
      }
    }
  }
  {
    \MyConsole
    {
      \MyCmd{f(2)}\\\relax
      [1] 1.94591
    }
  }
  最後のオブジェクトを返り値とする場合はreturnはなくても良い.
}

\MyFrame{\insertsection~}
{
  \MyExercise
  {
    複数行の関数を作成せよ.
  }
}

\section{入出力}

\MyFrame{\insertsection~テキストデータの出力}
{
  \MyDefinition{手順}
  {
    \texttt{write.csv}関数を使用して,オブジェクトデータを
    ファイルにCSVファイル形式*で書き込む.
    * CSV:カンマ区切
  }
  \MyConsole
  {
    \MyCmd{ d0 <- data.frame(name=c('panda','lion'),age=c(5,7)) }\\
    \MyCmd{ write.csv(d0, file = 'd0.csv') }
  }
  \texttt{quote = F}オプションをつけると
  出力ファイル内の文字列引用符「"」を削除できる.
  \texttt{write.csv(d0, file = 'd0.csv', quote = F)}
}

\MyFrame{\insertsection~テキストデータの出力}
{
  \MyExercise
  {
    データフレームを作成しファイルに出力せよ.
  }
}

\MyFrame{\insertsection~テキストデータの出力}
{
  \MyDefinition{手順}
  {
    \texttt{read.csv}関数を使用して,CSVファイルを読み込み
    オブジェクトに格納する.
  }
  \MyConsole
  {
    \MyCmd{d1 <- read.csv(file = 'd0.csv')}\\
    \MyCmd{str(d1)}
  }
  \texttt{stringsAsFactors = F}オプションを付加すると
  文字列の自動因子化を抑制する.\\
  \texttt{read.csv(file = 'd0.csv', stringsAsFactors = F)}
}

\MyFrame{\insertsection~テキストデータの出力}
{
  \MyExercise
  {
    CSVファイルを読み込み,オブジェクトに格納せよ.
  }
}

\MyFrame{\insertsection~\Excel データの入力}
{
  \MyDefinition{手順}
  {
    excel.linkパッケージを利用し,\R とリンクさせる.\\
    パッケージの利用コマンド: \texttt{library(excel.link)}
  }
  \MyConsole
  {
    \MyCmd{library(excel.link)}\\
    \MyCmd{xl.workbook.open('test.xlsx')}\hfill \MyCmt{\Excel を開く}\\
    \MyCmd{d <- data.frame(x=3:1, y=-1:1)}\\
    \MyCmd{xl['Sheet1!A1'] <- d}\hfill \MyCmt{Sheet1のA1を起点としてデータを書き込み}\\
    \MyCmd{xl['Sheet1!B2'] -> x}\hfill \MyCmt{Sheet1のB2からデータを読み込み}\\
    \MyCmd{xl.workbook.save('test.xlsx')}\hfill \MyCmt{Excel を保存}
  }
  \vspace{-2mm}
  \texttt{xlr}: 行名付き,
  \texttt{xlc}:列名付き,
  \texttt{xlrc}:行列名付き入出力
}

\MyFrame{\insertsection~テキストデータの出力}
{
  \MyExercise
  {
    パッケージのヘルプにあるサンプルコードを用いて\Excel を操作せよ.
  }
}

\end{document}
